\section{Command Line Options}\label{sec:options}

In this section, we briefly describe the meanings of
command line options supported by
\gringo\ (Section~\ref{subsec:opt:gringo}),
\clingo\ (Section~\ref{subsec:opt:clingo}),
\iclingo\ (Section~\ref{subsec:opt:iclingo}), and
\clasp\ (Section~\ref{subsec:opt:clasp}).
Each of these tools displays its available options
when invoked with flag \code{--help} or \code{-h},
respectively.%
\footnote{%
  Note that our description of command line options
  is based on Version~2.0.2 of \gringo, \clingo, and \iclingo\
  as well as Version~1.1.2 of \clasp.
  While it is rather unlikely that command line options will
  disappear in future versions,
  additional ones might be introduced.
  We will try to keep this document up-to-date,
  but checking the help information shipped 
  with a new version is always a good idea.}
The approach of distinguishing long options, starting with ``\code{--},''
and short ones of the form ``\code{-\textit{l}},''
where \code{\textit{l}} is a letter,
follows the GNU Coding Standards~\cite{GNUcoding}.
For obvious reasons,
short forms are made available only for the most common (long) options.
Some options, also called flags, do not take any argument,
while others require arguments.
An argument \code{\textit{arg}} is provided to a (long) option \code{\textit{opt}}
by writing 
``\code{--\textit{opt}=\textit{arg}}'' or
``\code{--\textit{opt} \textit{arg}},''
while only
``\code{-\textit{l} \textit{arg}}''
is accepted for a short option \code{\textit{l}}.
For each command line option,
we below indicate whether it requires an argument,
and if so, we also describe its meaning.


\subsection{\gringo\ Options}\label{subsec:opt:gringo}

An abstract invocation of \gringo\ looks as follows:
%
\begin{lstlisting}[numbers=none]
gringo [ options | filenames ]
\end{lstlisting}
%
Note that options and filenames do not need to be passed to \gringo\
in any particular order.
If neither a filename nor an option that makes \gringo\ exit (see below)
is provided, \gringo\ reads from the standard input.
In the following, we list and describe the options accepted by \gringo\
along with their particular arguments (if required):
%
\begin{description}
\item[\code{--help,-h}]~\\
Print help information and exit.
\item[\code{--version,-v}]~\\
Print version information and exit.
\item[\code{--syntax}]~\\
Print syntax information (about the input language) and exit.
\item[\code{--stats}]~\\
Print (extended) statistic information before termination.
\item[\code{--verbose,-V}]~\\
Print additional (progress) information during computation.
\item[\code{--debug}]~\\
Print internal representations of rules during grounding.
(This may be used to identify either semantic errors in an input program or
 performance bottlenecks.)
\item[\code{--const,-c \textit{c}=\textit{t}}]~\\
Replace occurrences (in the input program) of a constant~\code{\textit{c}}
with a term~\code{\textit{t}}.
\item[\code{--text,-t}]~\\
Output ground program in (human-readable) text format.
\item[\code{--lparse,-l}]~\\
Output ground program in \lparse's numerical format~\cite{lparseManual}.
\item[\code{--aspils,-a 1|2|3|4|5|6|7}]~\\
Output ground program in \emph{ASPils} format~\cite{gejaosscth08a},
where the argument specifies one of the seven normal forms
defined in~\cite{gejaosscth08a}.
\item[\code{--ground,-g}]~\\
Enable lightweight mode for processing a ground input program.
(This option is recommended to 
 omit unnecessary overhead if the input program is already ground,
 but it leads to a syntax error (cf.\ Section~\ref{subsec:error}) otherwise.)
\item[\code{--bindersplit yes|no}]~\\
Enable or disable binder-splitting~\cite{gescth07a} in the instantiation of rules.
(This option is included mainly for comparison purposes, and generally,
 it is recommended to keep the default argument value \code{yes}.)
\item[\code{--ifixed \textit{n}}]~\\
Use \code{\textit{n}} as fix step number if the input program
contains \const{\#cumulative} or \const{\#volatile} statements.
(This option permits the handling of programs written for \iclingo\
 in a traditional single pass computation.)
\item[\code{--ibase}]~\\
Process only the static part (that can be initiated by a \const{\#base} statement)
of an input program.
(This option may be used to investigate the basic setting of a problem
 including some mutable bound.)
\end{description}
%
The default command line when invoking \gringo\ is as follows:
%
\begin{lstlisting}[numbers=none]
gringo --lparse --bindersplit=yes
\end{lstlisting}
That is, \gringo\ usually outputs a ground program in \lparse's numerical format,
dealt with by various solvers (cf.\ Section~\ref{subsec:asp}),
and it performs binder-splitting in rule instantiation.


\subsection{\clingo\ Options}\label{subsec:opt:clingo}

ASP system \clingo\ combines grounder \gringo\ and solver \clasp\
via an internal interface.
An abstract invocation of \clingo\ looks as follows:
%
\begin{lstlisting}[numbers=none]
clingo [ number | options | filenames ]
\end{lstlisting}
%
A numerical argument is permitted for backward compatibility to
the usage of solver \smodels~\cite{siniso02a},
where it specifies the maximum number of answer sets to be computed
(\code{0}~standing for all answer sets).
As with \gringo, a number, options, and filenames do not need to be passed to \clingo\
in any particular order.
Given that \clingo\ combines \gringo\ and \clasp,
it accepts all options described in the previous section
and in Section~\ref{subsec:opt:clasp}.
In particular, (long) options \code{--help}, \code{--version}, and
\code{--syntax} make \clingo\ print the desired information and exit,
while \code{--text}, \code{--lparse}, and \code{--aspils}
instruct \clingo\ to output a ground program (rather than solving it) like \gringo.
If neither a filename nor an option that makes \clingo\ exit
(see Section~\ref{subsec:opt:gringo}) is provided, \clingo\ reads from the standard input.
Beyond the options described in Section~\ref{subsec:opt:gringo} and~\ref{subsec:opt:clasp},
\clingo\ has a single additional option:
%
\begin{description}
\item[\code{--clasp}]~\\
Run \clingo\ as a plain solver (using embedded \clasp).
\end{description}
%
Finally, the default command line when invoking \clingo\
consists of all \clasp\ defaults (cf.\ Section~\ref{subsec:opt:clasp})
and option \code{--bindersplit=yes} of \gringo.


\subsection{\iclingo\ Options}\label{subsec:opt:iclingo}

Incremental ASP system \iclingo\ extends \clingo\ by interleaving
grounding and solving for problems including a mutable bound,
and an abstract invocation of \iclingo\ is as with \clingo:
%
\begin{lstlisting}[numbers=none]
iclingo [ number | options | filenames ]
\end{lstlisting}
%
The external behavior of \iclingo\ is similar to \clingo,
described in the previous section, except for the fact that
option \code{--ifixed} is ignored by \iclingo\ if not run as a grounder
(via one of (long) options \code{--text}, \code{--lparse}, or \code{--aspils}).
However, option \code{--clingo} (see below) may be used to let \iclingo\
work like \clingo.
The additional options of \iclingo\ focus on customizing
incremental computations:
%
\begin{description}
\item[\code{--istats}]~\\
Print statistic information for each incremental solving step.
\item[\code{--imin \textit{n}}]~\\
Perform at least \code{\textit{n}} incremental solving steps before termination.
(This may be used to force steps regardless of the termination condition
 set via \code{--istop}.)
\item[\code{--imax \textit{n}}]~\\
Perform at most \code{\textit{n}} incremental solving steps before termination.
(This may be used to limit steps regardless of the termination condition
 set via \code{--istop}.)
\item[\code{--istop SAT|UNSAT}]~\\
Terminate after an incremental solving step 
in which some (\code{SAT}) or no (\code{UNSAT})
answer set has been found.
\item[\code{--iquery \textit{n}}]~\\
Start with incremental solving at step number~\code{\textit{n}}.
(This may be used to skip some solving steps,
 still accumulating static and cumulative rules for these steps.)
\item[\code{--ilearnt keep|forget}]~\\
Maintain (\code{keep}) or delete (\code{forget}) learnt constraints
in-between incremental solving steps.
(This option configures the behavior of embedded \clasp.)
\item[\code{--iheuristic keep|forget}]~\\
Maintain (\code{keep}) or delete (\code{forget}) heuristic information
in-between incremental solving steps.
(This option configures the behavior of embedded \clasp.)
\item[\code{--clingo}]~\\
Run \iclingo\ as a non-incremental ASP system (like \clingo).
\end{description}
%
As with \clingo, the default command line when invoking \iclingo\
consists of all \clasp\ defaults, explained in the next section,
option \code{--bindersplit=yes} of \gringo\ (cf.\ Section~\ref{subsec:opt:gringo}),
along with \code{--istop=SAT}, \code{--iquery=1}, \code{--ilearnt=keep}, and
\code{--iheuristic=forget}.
That is, incremental solving starts at step number~\code{1} and stops
after a step in which some answer set has been found.
In-between incremental solving steps, embedded \clasp\ maintains learnt constraints
but deletes heuristic information.


\subsection{\clasp\ Options}\label{subsec:opt:clasp}

Stand-alone \clasp~\cite{gekanesc07b} is a solver for ground programs in
\lparse's numerical format~\cite{lparseManual}.
An abstract invocation of \clasp\ looks as follows:
%
\begin{lstlisting}[numbers=none]
clasp [ number | options ]
\end{lstlisting}
%
As with \clingo\ and \iclingo,
a numerical argument specifies the maximum number of answer sets to be computed,
where \code{0}~stands for all answer sets.
(The number of requested answer sets can likewise be set via long option
 \code{--number} or its short form~\code{-n}.)
If neither a filename (via long option \code{--file} or its short form \code{-f})
nor an option that makes \clasp\ exit (see below) is provided,
\clasp\ reads from the standard input.
In fact, it is typical to use \clasp\ in a pipe with \gringo\ in the following way:
\begin{lstlisting}[numbers=none]
gringo [ ... ] | clasp [ ... ]
\end{lstlisting}
In such a pipe, \gringo\ instantiates an input program and outputs the ground
rules in \lparse's numerical format,
which is then consumed by \clasp\ that computes and outputs answer sets.
Note that \clasp\ offers plenty of options to configure its behavior.
We thus categorize them according to their functionalities in the following description.


\subsubsection{General Options}\label{subsec:clasp:general}

We below group general options of \clasp,
used to configure its global behavior.
%
\begin{description}
\item[\code{--help,-h}]~\\
Print help information and exit.
\item[\code{--version,-v}]~\\
Print version information and exit.
\item[\code{--stats}]~\\
Print (extended) statistic information before termination.
\item[\code{--file,-f \textit{filename}}]~\\
Read from file \code{\textit{filename}} (rather than from the standard input).
\item[\code{--number,-n \textit{n}}]~\\
Compute at most~\code{\textit{n}} answer sets,
$\code{\textit{n}}=\code{0}$ standing for compute all answer sets.
\item[\code{--quiet,-q}]~\\
Do not print computed answer sets.
(This is useful for benchmarking.)
\item[\code{--seed \textit{n}}]~\\
Use seed~\code{\textit{n}} (rather than~\code{1}) for random number generator.
\item[\code{--brave}]~\\
Compute the brave consequences (union of all answer sets) of a logic program.
\item[\code{--cautious}]~\\
Compute the cautious consequences (intersection of all answer sets) of a logic program.
\item[\code{--opt-all}]~\\
Compute all optimal answer sets (cf.\ Section~\ref{subsec:gringo:optimize}).
(This is implemented by enumerating~\cite{gekanesc07c} answer sets that are not worse
 than the best one found so far.)
\item[\code{--opt-restart}]~\\
Restart the search from scratch (rather than doing enumeration~\cite{gekanesc07c})
after finding a provisionally optimal answer set.
(This may speed up finding the optimum.)
\item[\code{--opt-value \textit{n1}[,\textit{n2},\textit{n3}...]}]~\\
Initialize objective function(s) to minimize with \code{\textit{n1}[,\textit{n2},\textit{n3}...]}.
Depending on the reasoning mode,
only equitable or strictly better answer sets are computed.
\item[\code{--supp-models}]~\\
Compute supported models~\cite{apblwa88a} (rather than answer sets).
\item[\code{--dimacs}]~\\
Compute models of a propositional formula in DIMACS/CNF format~\cite{dimacs}.
\item[\code{--trans-ext all|choice|weight|no}]~\\
Compile extended rules~\cite{siniso02a} into normal rules of form (\ref{eq:normal:rule}).
Arguments \code{choice} and \code{weight}
state that all ``choice rules'' or all ``weight rules,'' respectively,
are to be compiled into normal rules,
while \code{all} means that both and \code{no} that none of them
are subject to compilation.
\item[\code{--eq \textit{n}}]~\\
Run equivalence reasoning~\cite{gekanesc08a} for \code{\textit{n}} iterations,
$\code{\textit{n}}=\code{-1}$ and $\code{\textit{n}}=\code{0}$ standing for
run to fixpoint or do not run equivalence reasoning, respectively.
\item[\code{--sat-prepro yes|no|\textit{n1}[,\textit{n2},\textit{n3}]}]~\\
Run \emph{SatElite}-like preprocessing~\cite{eenbie05a}
for at most~\code{\textit{n1}} iterations 
($\code{\textit{n1}}=\code{-1}$ standing for run to fixpoint),
using cutoff~\code{\textit{n2}} for variable elimination
($\code{\textit{n2}}=\code{-1}$ standing for no cutoff), and
for no longer than~\code{\textit{n3}} seconds
($\code{\textit{n3}}=\code{-1}$ standing for no time limit).
Arguments \code{yes} and \code{no} mean
$\code{\textit{n1}}=\code{\textit{n2}}=\code{\textit{n3}}=\code{-1}$
(that is, run to fixpoint)
or that \emph{SatElite}-like preprocessing is not to be run at all, respectively.
\item[\code{--rand-watches yes|no}]~\\
Initially choose watched literals randomly (with argument \code{yes})
or systematically (with argument \code{no}).
\end{description}
%
Having introduced the general options of \clasp,
let us note that the options below \code{--dimacs} in the above list
are quite low-level and more or less an issue of fine-tuning.
More important is the fact that virtually all optimization functionalities
are only provided by \clasp\ if the maximum number of answer sets to be
computed is set to~\code{0} (standing for all answer sets),
as it is likely to stop search too early otherwise.
The same applies to computing either brave or cautious consequences
(via one of the flags \code{--brave} and \code{--cautious}).


\subsubsection{Search Options}\label{subsec:clasp:search}

The options listed below can be used to configure the main search
strategies of \clasp.
%
\begin{description}
\item[\code{--lookback yes|no}]~\\
Enable or disable lookback techniques (cf.\ Section~\ref{subsec:clasp:back}).
(This option is included mainly for comparison purposes, and generally,
 it is recommended to keep the default argument value \code{yes}.)
\item[\code{--lookahead atom|body|hybrid|auto|no}]~\\
Apply failed-literal detection~\cite{freeman95}
to atoms (with argument \code{atom}),
to rule bodies (with argument \code{body}),
to atoms and rule bodies like in \nomorepp~\cite{angelinesc05c} (with argument \code{hybrid}),
or let \clasp\ pick one of the previous three possibilities
based on program properties (with argument \code{auto}).
Failed-literal detection is switched off via argument~\code{no}.
\item[\code{--initial-lookahead}]~\\
Apply failed-literal detection (to atoms) in preprocessing,
but not (necessarily) during search.
\item[\code{--heuristic Berkmin|Vmtf|Vsids|Unit|None}]~\\
Use \emph{BerkMin}-like decision heuristic~\cite{golnov02a}
(with argument~\code{Berkmin}),
\emph{Siege}-like decision heuristic~\cite{ryan04a}
(with argument~\code{Vmtf}),
\emph{Chaff}-like decision heuristic~\cite{momazhzhma01a}
(with argument~\code{Vsids}),
\emph{Smodels}-like decision heuristic~\cite{siniso02a}
(with argument~\code{Unit}),
or (arbitrary) static variable ordering
(with argument~\code{None}).
\item[\code{--rand-freq \textit{p}}]~\\
Perform random (rather than heuristic) decisions
with probability $0\leq\code{\textit{p}}\leq 1$.
\item[\code{--rand-prob yes|no|\textit{n1},\textit{n2}}]~\\
Run \emph{Satzoo}-like random probing~\cite{een03a},
initially performing~\code{\textit{n1}} passes of up to~\code{\textit{n2}} conflicts
making random decisions.
Arguments \code{yes} and \code{no} mean
$\code{\textit{n1}}=\code{50},\linebreak[1]\code{\textit{n2}}=\code{20}$
or that random probing is not to be run at all, respectively.
\end{description}
%
Let us note that switching the above options can have dramatic effects
(both positively and negatively) on the search performance of \clasp.
Sticking to the default argument values \code{yes} for \code{--lookback}
and \code{no} for \code{--lookahead} is generally recommended.
If nonetheless performance bottlenecks are observed,
it is worthwhile to give \code{Vmtf} and \code{Vsids} for \code{--heuristic}
a try, in particular,
when the program under consideration is huge 
but scarcely yields conflicts during search.


\subsubsection{Lookback Options}\label{subsec:clasp:back}

The following options have an effect only if \code{--lookback}
is turned on, that is, its argument value must be \code{yes}.
%
\begin{description}
\item[\code{--restarts,-r \textit{n1}[,\textit{n2},\textit{n3}]|no}]~\\
Choose and parameterize a restart policy.
If a single argument \code{\textit{n1}} is provided,
\clasp\ restarts search from scratch after a number of conflicts
determined by a universal sequence~\cite{lusizu93a}, where \code{\textit{n1}}
constitutes the base unit.
If two arguments \code{\textit{n1},\textit{n2}} are specified,
\clasp\ runs a geometric sequence~\cite{eensor03a},
restarting every $\code{\textit{n1}}*\code{\textit{n2}}^i$ conflicts,
where~$i$ is the number of restarts performed so far.
Given three arguments \code{\textit{n1},\textit{n2},\textit{n3}},
\clasp\ repeats geometric restart sequence $\code{\textit{n1}}*\code{\textit{n2}}^i$
when it reaches an outer limit~$\code{\textit{n1}}*\code{\textit{n3}}^j$~\cite{biere08a},
where~$j$ counts how often the outer limit has been hit so far.
Finally, restarts are disabled via argument~\code{no}.
\item[\code{--local-restarts}]~\\
Count conflicts locally~\cite{ryvstr08a} (rather than globally)
for deciding when to restart.
\item[\code{--bounded-restarts}]~\\
Perform bounded restarts in answer set enumeration~\cite{gekanesc07c}.
\item[\code{--save-progress}]~\\
Use cached (rather than heuristic) decisions~\cite{pipdar07a} if available.
\item[\code{--shuffle,-s \textit{n1},\textit{n2}}]~\\
Shuffle internal data structures after~\code{\textit{n1}} restarts
($\code{\textit{n1}}=\code{0}$ standing for no shuffling)
and then reshuffle every~\code{\textit{n2}} restarts 
($\code{\textit{n2}}=\code{0}$ standing for no reshuffling).
\item[\code{--deletion,-d \textit{n1}[,\textit{n2},\textit{n3}]}]~\\
Limit the number of learnt constraints to 
$\min\{(c/\code{\textit{n1}})*\code{\textit{n2}}^i,c*\code{\textit{n3}}\}$,
where~$c$ is the initial number of constraints and~$i$ is the number of restarts
performed so far.
\item[\code{--reduce-on-restart}]~\\
Delete a portion of learnt constraints after every restart.
\item[\code{--strengthen bin|tern|all|no}]~\\
Check binary (with argument \code{bin}),
binary and ternary (with argument \code{tern}), or 
all (with argument \code{all}) antecedents for
self-subsumption~\cite{eenbie05a}
in order to strengthen a constraint to learn.
Strengthening is disabled via argument~\code{no}.
\item[\code{--recursive-str}]~\\
Recursively apply strengthening, as proposed in~\cite{soreen05a}.
\item[\code{--loops common|distinct|shared|no}]~\\
Learn loop nogood~\cite{gekanesc07a} per atom in an unfounded set~\cite{gerosc91a}
(with argument \code{common}),
shrink unfounded set before learning another loop nogood
(with argument \code{distinct}),
learn loop formula~\cite{linzha04a} for atoms in an unfounded set
(with argument \code{shared}),
or do not record unfounded sets at all
(with argument \code{no}).
\item[\code{--contraction \textit{n}}]~\\
Temporarily truncate learnt constraints over more than \code{\textit{n}} variables~\cite{ryan04a}.
\end{description}
%
As with search options described in the previous section,
switching lookback options may have a significant impact on the
performance of \clasp.
In particular, we suggest trying the universal restart sequence~\cite{lusizu93a}
with different base units~\code{\textit{n1}} or even disabling restarts
(both via (long) option \code{--restarts} or its short form~\code{-r})
in case that performance needs to be improved.

Finally, let us consider the default command line of \clasp:
%
\begin{lstlisting}[numbers=none]
clasp 1 --seed=1 --trans-ext=no --eq=5 --sat-prepro=no
        --rand-watches=yes --lookback=yes --lookahead=no
        --heuristic=Berkmin --rand-freq=0.0 --rand-prob=no
        --restarts=100,1.5 --shuffle=0,0 --deletion=3.0,1.1,3.0
        --strengthen=all --loops=common --contraction=250
\end{lstlisting}
%
Considering only the most significant defaults,
numeric argument~\code{1} instructs \clasp\ to terminate immediately
after finding an answer set,
and with \code{--lookback=yes}, \code{--restarts=100,1.5} lets \clasp\ apply a geometric
restart policy.

%%% Local Variables: 
%%% mode: latex
%%% TeX-master: "guide"
%%% End: 
