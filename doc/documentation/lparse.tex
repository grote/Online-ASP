\section{Differences to the Language of \lparse}\label{sec:lparse}

We below provide a (most likely incomplete) list of differences between the
input languages of \gringo~\cite{gescth07a} and \lparse~\cite{lparseManual}.
First of all, it is important to note that the language of \gringo\
significantly extends the one of \lparse.
For instance, \const{min} and \const{max} aggregates (cf.\ Section~\ref{subsec:gringo:aggregate})
are not supported by \lparse.
Furthermore, \gringo\ provides full support for variables within compound terms
(over uninterpreted functions with non-zero arity).
That is, an atom like \code{p(\const{f}(\var{X}))} in the positive body of a rule
can potentially be used to bind~\var{X} (cf.\ Section~\ref{subsec:lambda}),
while \lparse\ would treat \code{\const{f}(\var{X})} like an arithmetic
function (cf.\ Section~\ref{subsec:gringo:arith}) whose variables cannot be bound.
Finally, \gringo\ deals with level-restricted programs (cf.\ Section~\ref{subsec:lambda}),
while \lparse\ requires programs to be $\omega$-restricted~\cite{syrjanen01a}.
As the latter are more restrictive,
programs for \lparse\ tend to be more verbose than the ones for \gringo.
Thus, we do not suggest writing programs in the input language of \gringo\
for compatibility to \lparse.

However, a bulk of existing encodings are written for \lparse\
(see, e.g.,~\cite{asparagus}),
and \gringo\ (likewise, \clingo\ and \iclingo) should actually be able to deal with most of them.
If this is not case, one of the following might be the reason:
%
\begin{itemize}
\item
The input contains primed atoms like \code{\pred{p}'}, which are (currently) not supported by \gringo.
\item
Symbolic names are used for built-in constructs, e.g., \const{plus} or \pred{eq}
for built-in arithmetic function~\code{+} or predicate~\code{==}, respectively.
With a few exceptions (e.g., \const{abs} introduced in Section~\ref{subsec:gringo:arith}),
such names are not associated to built-in constructs by \gringo,
as they may accidentally clash with users' names otherwise.
\item
The input contains (or is instantiated to) 
a \const{count} aggregate (or a cardinality constraints, respectively)
containing duplicates of literals.
Such duplicates are not removed by \lparse, e.g.,
it treats \code{2\char`\{\pred{p}(\const{c}),\pred{p}(\const{c})\char`\}}
as a synonym for \code{2[\pred{p}(\const{c})=1,\pred{p}(\const{c})=1]}.
In contrast, \gringo\ associates curly brackets to a set
(and square brackets to a multiset), as described in Section~\ref{subsec:gringo:aggregate}, so that
\code{2\char`\{\pred{p}(\const{c}),\pred{p}(\const{c})\char`\}} is the same as
\code{2\char`\{\pred{p}(\const{c})\char`\}},
where the latter can clearly not hold.
\item
Pooling is expanded differently,
e.g.,
\lparse\ interprets
\code{\pred{p}(\var{X},\var{Y};\var{X},\var{Z})} 
as a shorthand for
\code{\pred{p}(\var{X},\var{Y}),\pred{p}(\var{X},\var{Z})},
while \gringo\ expands it to
\code{\pred{p}(\var{X},\var{Y},\var{Z}),\linebreak[1]\pred{p}(\var{X},\var{X},\var{Z})},
as explained in Section~\ref{subsec:gringo:pool}.
\item
The input contains some (rather unusual) meta-statement, e.g.,
\code{\#external}, not supported by \gringo.
\end{itemize}
%
As indicated above, the provided list is probably incomplete.
If you would like some difference to be added,
please contact the authors of this guide.


%%% Local Variables: 
%%% mode: latex
%%% TeX-master: "guide"
%%% End: 
